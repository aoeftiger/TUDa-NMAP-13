\pdfobjcompresslevel=0
\documentclass[a4paper,graphics,t]{beamer} %, handout

\usepackage[utf8]{inputenc}
%\usepackage[scaled]{uarial} % not available in Linux
\renewcommand*\familydefault{\sfdefault} %% Only if the base font of the document is to be sans serif

\usepackage{multirow}
\usepackage[T1]{fontenc}
% \usepackage{titling}

\title{\texorpdfstring{%
    %\\[0.5cm]
    Lecture 13: Wrap-up \\[0.3em] Numerical Methods of \break Accelerator Physics %
    }{%
        Lecture 13: Wrap-up Numerical Methods of Accelerator Physics%
    }%
}
\subtitle{\texorpdfstring{%
        \normalfont{Dr.\ Adrian Oeftiger}%
    }{}%
}
\date{3 February 2023}
\author{Dr.\ Adrian Oeftiger}

\usetheme{FAIR}
    \setbeamercovered{invisible} %default
    % \setbeamercovered{transparent} % Items to be uncovered already visible, but almost transparent

\addbibresource{resources.bib}
\usepackage{relsize}

\captionsetup{font=scriptsize,labelfont={bf,sf}}

\begin{document}

\begin{frame}[noframenumbering]{}
    \titlepage
\end{frame}

\begin{frame}[c]{Overview}
    Learning objects for lecture series:
    \begin{itemize}
        \item basic models of accelerator physics
        \item suitable methods for their implementation
    \end{itemize}
    \vspace{1em}
    Topics covered:
    \begin{enumerate}
        \item \textbf{reduced models}: steering, focusing, acceleration
        \item \textbf{maps} of linear periodic systems + \textbf{stability analysis}
        \item beam \textbf{transport models}: particles, beam distributions (self-consistent modelling)
        \item \textbf{control-room} applications and \textbf{diagnostics}:
        \begin{itemize}
            \see tune reconstruction
            \see tomographic reconstruction of phase space
            \see closed-orbit control
        \end{itemize}
    \end{enumerate}
\end{frame}

\begin{frame}{Examination}
    \begin{itemize}
        \item \textbf{30 min} oral exam in TEMF library on \textbf{14 April 2023} \break (check your TUCaN for the time slot or ask me)
        \item format $=$ conceptual discussion on models of accelerator physics and numerical implementations
        \item exam material $=$ $\sum$ summary slides
        \attention you do \textbf{not} need to know how to write python code -- \break focus on the ideas and concepts we discussed!
    \end{itemize}
\end{frame}

\begin{frame}[t]\frametitle{Golden Thread}
    \begin{enumerate}[I.]
        \item \textbf{basic concepts}: \textbf{\textcolor{blue}{lectures 1-3}}
        \only<2>{
        \begin{itemize}
            \sothat using the simple pendulum as example
            \see time scales in a synchrotron \break (transverse / longitudinal motion period, storage times)
            \see phase space (system state), Hamiltonian (equations of motion)
            \see discrete integrators: Euler, Euler-Cromer, leapfrog
            \see statistical moments, emittance
            \see non-linearities, Liouville theorem vs.\ filamentation (emittance growth)
            \see discrete frequency analysis, NAFF algorithm (vs.\ FFT)
            \see control of simulation error sources:
            \begin{enumerate}[1.]
                \item discretisation error (symplecticity!)
                \item modelling error
                \item numerical artefacts
                \item (input error)
            \end{enumerate}
            \see deterministic chaos, \break early indicators: (max.) Lyapunov exponent, frequency map analysis
        \end{itemize}
        }
        \item<1,3-> \textbf{longitudinal dynamics}: \textbf{\textcolor{blue}{lectures 4-6}}
        \only<3>{
        \begin{itemize}
            \see Lorentz force, electric longitudinal field $E_z$ to accelerate
            \see beam rigidity, paraxial approximation
            \see momentum compaction, phase slippage, transition energy
            \see phase focusing and stability (classical vs.\ relativistic regime)
            \see longitudinal tracking equations (discrete one-turn map)
            \see synchrotron Hamiltonian, rf bucket
            \see Monte-Carlo sampling (random number generation)
            \see equilibrium distributions (thermal PDF), \break small-amplitude approximation vs.\ nonlinear matching
            \see emittance growth mechanisms (filamentation $\leftrightarrow$ bucket non-linearity) from dipole and quadrupole moment oscillations
        \end{itemize}
        }
        \item<1,4-> \textbf{transverse dynamics}: \textbf{\textcolor{blue}{lectures 7-8}}
        \only<4>{
        \begin{itemize}
            \see magnetic fields for bending (steering) and focusing
            \see multipole representation, dipole / quadrupole / sextupole magnets
            \see Hill differential equation, quasi-harmonic oscillation
            \see betatron transport matrices
            \see FODO cell, alternate-gradient focusing
            \see optics / Twiss functions, $\beta$-function as beam envelope, \break dispersion function
            \see stability of periodic transport maps
            \see betatron tune, chromaticity
        \end{itemize}
        }
        \item<1,5-> \textbf{applications}: %\textbf{\textcolor{blue}{lectures 9-14}}
        \only<5>{
        \begin{itemize}
            \see longitudinal phase-space tomography \textbf{\textcolor{blue}{(lecture 9)}}
            \begin{itemize}
                \item Radon transform, sinogram, Fourier Slice Theorem
                \item filtered back projection %(inverse Radon transform) \break
                vs.\ algebraic reconstruction technique
            \end{itemize}
            \see closed orbit distortion \textbf{\textcolor{blue}{(lecture 10)}}
            \begin{itemize}
                \item local orbit correction (bumps)
                \item global orbit correction (orbit response matrix, SVD)
            \end{itemize}
            \see self-consistent modelling / collective effects \textbf{\textcolor{blue}{(lecture 12)}}
            \begin{itemize}
                \item categories of beam interactions (space charge, ...)
                \item longitudinal space charge, line density derivative $\lambda'$ model
                \item microwave instability
            \end{itemize}
            \see machine learning \textbf{\textcolor{blue}{(lectures 11, 14)}}
            \begin{itemize}
                \see reinforcement learning (discrete vs.\ continuous state/action spaces)%, Q-learning \& actor-critic methods)
                \see Bayesian optimisation (Gaussian processes, uncertainty modelling)
            \end{itemize}
        \end{itemize}
        }
    \end{enumerate}
    \vspace{0.5em}
\end{frame}

\begin{frame}[plain, noframenumbering, c]
    % \vspace{2cm}
    \begin{beamercolorbox}[sep=12pt,center]{block title}
        \usebeamerfont{section title}
        You have now solid fundamental knowledge on numerical modelling of periodic physics + have seen in action a few dynamical examples from accelerator physics! \\[1em]
        ... and perhaps became a happy python user. \\[1em]
        Well done! :))
    \end{beamercolorbox}
\end{frame}

% \newcounter{finalframe}
% \setcounter{finalframe}{\value{framenumber}}

% % \section{References}
% % \let\frametitle\oldframetitle
% \begin{frame}[allowframebreaks]{References}
%    \begin{spacing}{1.2}
%    \printbibliography[heading=none]
%    \end{spacing}
% \end{frame}

% \setcounter{framenumber}{\value{finalframe}}

\end{document}
